\documentclass[UTF8]{ctexart}
\usepackage{amsmath}
\usepackage{graphicx}
\usepackage{hyperref}

\title{Modbus RTU Learning based on RS485}
\author{倪煜晖}
\date{\today}

\begin{document}

\maketitle

% 生成目录
\tableofcontents

\section{说明}
记录Modbus RTU通讯协议学习过程,用于实现xArm6机械臂与知行机器人夹爪之间的通信,以备后续查看使用。

\section{Modbus RTU与 RS485初探}

\subsection{简介}
Modbus是一种应用协议,RTU是一种通信模式,而RS485是总线串行标准。前二者工作在应用层与链路层,而RS485工作在物理层。

\subsection{联系与区别}
\begin{itemize}
    \item Modbus RTU:一种主从通信协议。它定义了数据传输的规则,包括数据帧的格式、帧的开始和结束标志、地址域、功能码、数据区和错误检测域等。例如,在一个 Modbus RTU 帧中,地址域用于标识从设备的地址,功能码用于指定主设备希望从设备执行的操作,如读取寄存器、写入寄存器等。
    \item RS485:一种电气接口标准,它规定了数据传输的物理层特性,如信号电平、传输速率、传输距离等。RS - 485 支持多点通信,能够在长距离和高噪声环境下可靠地传输数据。
\end{itemize}

在我们的任务中,RS - 485 提供了硬件层面的通信通道,Modbus RTU 则是在这个通道上运行的协议,规定了数据的传输格式、帧结构等内容,二者相互配合来实现设备之间的通信。

\subsection{重点}
RS485使用差分传输模式,使用双绞线$A,B$之间的电位差来实现通信。它的核心是一个主机与多个从机的通讯。这里需要注意的是,这和$\textbf{I/O}$通信完全没有关系,也就是说,我们$\textbf{I/O}$的五根线大概是没用的。由于RS485协议对电位敏感,建议在之后断开对这五根线的连接,保证接地唯一。


% 插入图片示例
\begin{figure}[htbp]
    \centering
    \includegraphics[width=0.8\textwidth]{master}
    \caption{一主多从}
    \label{ms}
\end{figure}
确认接线之后,我们把注意力更多放在Modbus RTU上。


\section{使用机器码实现Modbus RTU通信}


\end{document}